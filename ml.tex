% proposal.tex
\documentclass[main.tex]{subfiles}
\begin{document}
\chapter{Prediction of Mechanical Properties through Machine Learning} \label{ch:ml}

\section{Foreword} \label{sec:fw_ml}

The use of AM technologies to produce small batches of highly customized, complex parts in a reduced development cycle results extremely attractive to all industries. However, for AM parts to be fully adopted in industrial scenarios, engineers have to be able to confidently assess the structural integrity of the finished part under its intended loading conditions. This requirement is unfortunately not fully possible at the time this work was produced, partly because the mechanical properties of AM tend to be anisotropic, and partly because the relationships that exist between processing parameters, underlying physics of the process, and final mechanical part properties aren't fully comprehended. However, these obstacles present an interesting case for the application of Machine Learning (ML) techniques, where the inputs and outputs of a particular phenomenon are known, but there's a lack of explicit rules that indicate a relationship between the two. 

This work uses the Material Extrusion (ME) process as a case study for the application of ML techniques to predict the final mechanical properties of a printed part. Experimental work involved producing a variety of tensile coupons, developed under various printing conditions, and where the filament extrusion speed, filament extrusion force, and printing temperature were measured in real time using machines fitted with in-line sensors. These specimens were then tested up to tensile failure, and the collective data of printing parameters, measured process indicators, and mechanical tests results were used to train a Neural Network capable of predicting the tensile failure stress.

In the context of this dissertation, this represents an alternative method for part failure prediction to construction and evaluation of a failure envelope. However, it should be noted that both methods are not mutually exclusive, and as will be discussed, the author believes they can, and should, be combined1. 

\section{Introduction} \label{sec:ml_intr}

The set of printing conditions that lead to an optimal part in terms of mechanical properties are not fully comprehended due to the nuances associated with the interacting effects of the processing conditions, material behavior, paired with a commonplace lack of standardization in the field of AM as a whole. However, recent advancements in processing power and algorithms have made it easier than ever to deploy Machine Learning (ML) solutions, and the intricacies of the processing-properties relationships of AM techniques represent an interesting case for development of a ML system. These excel in cases where the inputs and outcomes of a particular phenomena or task are known, but connecting the two through an explicit set of rules or relationships can result extremely complex and time consuming \cite{Chollet2018}. In this manner, ML models are \emph{trained}, as opposed to explicitly programmed, as illustrated in Figure \ref{fig:MLvsP}, where the differences between ML and traditional programming philosophies are compared. 

\begin{figure}[!htbp]
	\center
	\includegraphics[height=6cm]{ML}
	\caption{Differences between traditional programming and machine learning. \cite{Chollet2018}} \label{fig:MLvsP_2}
\end{figure}

The potential to apply ML solutions in the field of AM has been noted by several authors \cite{Razvi2019,Meng2020}. Example cases include design-recommendation systems, topology optimization solutions, tolerancing and manufacturability assessment, and material classification and selection \cite{Razvi2019}. The specific algorithm applied for each case varied wildly depending on the nature of the task, but in general, Support Vector Machines (SVM) and Neural Networks (NN) appear to be the most prevalent solutions.

A Neural Network (NN) algorithm is effectively a facsimile of how biological neurons establish connections and communications with each other. In summary, the inputs of the problem are fed to a layer of nodes, or "neurons". Each node has itself a variety of connections to other neurons, and an associated weight and activation threshold, which if surpassed, triggers information transfer to its connections in subsequent layers. Finally, the information reaches the network stratus that estimates the outcome of whatever phenomena the model is trying to characterize, traditionally named the output layer \cite {Chollet2018, IBMCloudEducation2020, Geron2019}. The weights and activation thresholds of each node are iteratively tuned as the NN architecture is exposed to a training data set, while also being compared to a separate set of data points used for validation. Once the accuracy of the model reaches its desired value, the underlying communication between the neurons is capable of making predictions based on what the input layer is perceiving. A schematic of a NN can be seen in Figure \ref{fig:NN}. The particular NN shown in this image is called a Deep Neural Network, as the number of layers of nodes surpasses three \cite{IBMCloudEducation2020}. This type of architecture tends to be reserved for computationally complex tasks, such as text recognition or image processing.  

\begin{figure}[!htbp]
	\center
	\includegraphics[width=0.8\linewidth, keepaspectratio]{NN_scheme}
	\caption{Schematic of a NN \cite{IBMCloudEducation2020}} \label{fig:NN}
\end{figure}

The capability of NNs to model complex behaviors is rooted in the mathematical operations that happen behind the scenes. Each neuron behaves effectively as its own mini linear regression model, represented in Equation \ref{eq:neuron}. Here $X_{i}$ and $W_{i}$ represent one of the node's $m$ inputs and its associated weight respectively.  

\begin{equation} \label{eq:neuron}
	\sum_{i=1}^{m} W_{i}X_{i} + bias = z
\end{equation}  

The weighed sum of the inputs can then be used as is, or passed through an activation function. This signals the generation of an output that can then be used at face value, or transmitted to subsequent nodes if a threshold is surpassed. Assuming for the purposes of this example that the threshold is zero, and the activation function is the Heaviside step function, the output of a neuron can be computed as:

\begin{equation} \label{eq:heav}
	Heaviside(z) = \begin{cases}
		1 & \text{if } z \geq 1\\
		0 & \text{otherwise}
	\end{cases}
\end{equation}

Concatenating nodes in a forward fashion creates the concept of levels, or \emph{layers} in a network. When all neurons in a layer are fully connected to the nodes in the previous level, this is typically named a \emph{Dense} layer. Arranging more than one dense layer in series results in a NN \cite{Chollet2018, Geron2019}. 

The weights of each neuron are iteratively tuned in a process that involves penalizing the model using a loss function, that compares the predictions of the model with true output values using example data. This process is effectively an optimization task where the goal is to minimize the loss function. A schematic of the process can be seen in Figure \ref{fig:NN_it}, using a two layer network architecture as an example.

\begin{figure}[!htbp]
	\center
	\includegraphics[width=0.7\linewidth, keepaspectratio]{weight_ML}
	\caption{Iterative process of NN parameter tuning \cite{Chollet2018}} \label{fig:NN_it}
\end{figure}

Given the factors outlined this far, the fundamental goal of this research is to predict ME part mechanical performance by finding relations between processing conditions and strength through the use of sensors and machine learning. The success of this project would allow design engineers to confidently assess if a part manufactured through ME will meet the mechanical requirements imposed by its intended application. This tool can then be used to predict final mechanical properties of the part based on the data generated during the print. The features selected as controlled variables for the Design of Experiments (DoE) were Layer Height (LH), Nozzle Diameter (ND), and Print Speed (PS). These parameters were chosen based on previous research that shows that these slicing parameters had a tangible impact upon the final tensile strength of ME coupons \cite{Koch2017, Rankouhi2016}. Coupons are to be printed in both $0\deg$ and $90\deg$ orientations so the model can predict both the highest and lowest possible mechanical properties of each particular printing condition. Additional attention will be paid to the changes in required print force as these parameters are varied to produce the mechanical test coupons.

\section{Experimental Methods} \label{sec:ml_meth}

%\subsection{Feature Selection and Engineering} \label{ssec:ml_fs}
\subsection{Equipment and methods}\label{ssec:datag}

A set of 4 identical customized ME 3D printers (Minilab by FusedForm, Colombia) fitted with sensors capable of recording the force exerted by the filament upon the nozzle, discrete measurements of temperature, and changes in the extruded length over time were used to reproduce a variety of tensile coupons. A schematic of the printer setup can be seen in Figure 
\ref{fig:print_setup}. The data was collected using an Arduino board sampling at a frequency of 5 Hz, connected to MATLAB for visualization, processing, and logging. A Buttersworth filter was applied to amplify the signal-to-noise ratio of the outputs of the system. The code for the data acquisition and filtering can be found in Appendices \ref{ch:com} and \ref{ch:daq}.

\begin{figure}[!htbp]
	\center
	\includegraphics[height=7cm]{forcesetup}
	\caption{Schematic of modified ME printer with sensors} \label{fig:print_setup}
\end{figure}

Each print consisted of four rectangular tensile coupons of dimensions $25 \text{ mm}$ by $100 \text{ mm}$ by $3.2 \text{ mm}$, chosen to strike a balance between being relatively quick to print, having at least $50\text{ mm}$ of gauge length, and fitting in the jaws of the tensile testing equipment. Each coupon was printed in a part-by-part manner (as opposed to having a single layer of the print construct a slice of all coupons), to approximate real printing behavior as much as possible. 

A single experimental run yielded four replicates, which required post-processing to separate the force-data-speed pairings for each specimen. A pause was introduced between each part that would allow discerning when one specimen print was finished, and the next started. A schematic of the process is shown in Figure \ref{fig:print_dia}. 


\begin{figure}[!htbp]
	\center
	\includegraphics[height=6cm]{coupon_print_diagram}
	\caption{Schematic of print experiment} \label{fig:print_dia}
\end{figure}

Additionally as an exploratory experiment, the geometric information of the filament was collected and paired to the rest of the information stemming from the in-line measurements of a handful of prints, to assess if variations in the filament geometry resulted in notable changes in the required print force. This data was attained through the use of a laser micrometer and a conveyor belt, pulling the material at a constant, known speed. The process yielded discrete measurements of the filament diameter and ovality as a function of filament length and time. A schematic of the process can be seen in Figure \ref{fig:FD}.  

\begin{figure}[!htbp]
	\center
	\includegraphics[width=0.6\linewidth]{filament_measurement}
	\caption{Filament geometry information, acquired through a laser micrometer } \label{fig:FD}
\end{figure}

Finally, tensile testing was performed using an Instron 5967 dual column universal testing machine, fitted with a 30 kN load cell. All data acquisition was handled through the accompanying Instron Bluehill 3 software. A movement speed of 5 mm/min was used to deform the 50 mm gage section of the specimens, with all deformations being logged using an extensometer. To protect the samples from excessive gripping force, emery cloth tabs were used \cite{Capote2017}. 

\begin{figure}[h]
	\center
	\subfloat[Effect of diameter on measured filament speed \label{fig:D_sp}]{%
		\includegraphics[width=0.8\linewidth, keepaspectratio]{speed-OD}
	}
	\linebreak
	\subfloat[Effect of diameter on measured filament force \label{fig:D_f}]{%
		\includegraphics[width=0.8\linewidth, keepaspectratio]{force-OD}
	}
	\caption{Effect of diameter on filament force and speed} \label{fig:dia_f_sp}
\end{figure}

\begin{figure}[h]
	\center
	\subfloat[Effect of ovality on measured filament speed \label{fig:O_sp}]{%
		\includegraphics[width=0.8\linewidth, keepaspectratio]{speed-ovality}
	}
	\linebreak
	\subfloat[Effect of ovality on measured filament force \label{fig:O_f}]{%
		\includegraphics[width=0.8\linewidth, keepaspectratio]{force-ovality}
	}
	\caption{Effect of diameter on filament force and speed} \label{fig:ov_f_sp}
\end{figure}

\subsection{ML system architecture, training, and validation}\label{ssec:MLA}

The following step of this work would involve using small subsets of the training data to test multiple models and algorithms in a reasonable amount of time. Performance metrics such as the Mean Square Error (MSE) or the Mean Absolute Error (MAE) would help narrow down the optimal candidate for each task \cite{Geron2019}. Depending on the outcome, the final architecture of the predictive system will be decided, including the algorithms for each segment of the machine learning pipeline if applicable. Ultimately, the final architecture of the system will be trained using the training data, and benchmarked against the validation set to check for inherent issues to the ML field, such as overfitting, and to assess the validity of the predicted outcome. The programming language of choice will be \emph{Python 3}, given its relative ease of syntax, open-source nature, as well as the availability of data science and ML libraries and resources such as \emph{NumPy, pandas, and TensorFlow}.


%______________________________________________________________________________________________
% Nomenclature introduced in this chapter:
\nomenclature[A]{ML}{Machine Learning}% 
\nomenclature[A]{SVM}{Support Vector Machines}%
\nomenclature[A]{NN}{Neural Network}%
\nomenclature[A]{MSE}{Mean Square Error}%
\nomenclature[A]{MAE}{Mean Absolute Error}%

% Symbols introduced in this chapter:
%\nomenclature[S]{$X_t$}{Tensile strength in the 1-1 direction \nomunit{$MPa$}}
\end{document}