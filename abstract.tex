% abstract

\documentclass[main.tex]{subfiles}
\begin{document}
\setcounter{page}{1}
\chapter*{Abstract}%TODO
Fused Filament Fabrication (FFF) is arguably the most widely available Additive Manufacturing technology at the moment. Offering the possibility of producing complex geometries in a compressed product development cycle and in a plethora of materials, it comes as no surprise that FFF is attractive to multiple industries, including the automotive and aerospace segments. However, the high anisotropy of parts developed through this technique imply that part failure prediction is extremely difficult \textemdash a requirement that must be satisfied to guarantee the safety of the final user.

This dissertation explores two major strategies to solve this issue. The first involves predicting part failure through the use of Failure Criteria, allowing design engineers to assess if their part will be structurally sound given the mechanical constraints of the intended application. A limitation is that a failure envelope requires destructively testing a large number of samples produced under fixed manufacturing conditions, implying that once a failure envelope is calculated, it is only valid for whatever printing conditions were used to produce the test parts. This is an issue, as AM parts are highly sensitive to manufacturing parameters. 

The second strategy explored in this work is a direct attempt to solve this pain point. A Machine Learning solution that is capable of predicting how mechanical properties change as a function of printing parameters and in-line measurements was developed. The process required the use of unique FFF printers fitted with force and filament speed sensors. Given the unique capabilities of this FFF machine, the author tangentially made observations that can lead to improvements to existing analytical models that predict rheological behavior of the polymer melt within the nozzle of the printer. 

The results of this work encourage future research endeavors that combine the predictive capabilities of Machine Learning with Failure Criteria, and nudges the reader to the possibility of deployment of an FFF control system capable of autocorrecting the throughput of material to avoid part defects should a discrepancy between expected extrusion force, and measured values be detected.  
 
\vspace{10mm} %10mm vertical space
\textbf{Keywords:} FFF, ME, Failure Criteria, Machine Learning, Neural Networks.

\vfill %Send copyright notice to bottom of the page
\begin{center}
Copyright~\textcopyright: Gerardo A. Mazzei Capote (2021)

\emph{All rights reserved}	
\end{center}
\end{document}