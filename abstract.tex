% abstract

\documentclass[main.tex]{subfiles}
\begin{document}
\setcounter{page}{1}
\chapter*{Abstract}
Fused Filament Fabrication (FFF) is arguably the most widely available Additive Manufacturing technology at the moment. Offering the possibility of producing complex geometries in a compressed product development cycle and in a plethora of materials, it comes as no surprise that FFF is attractive to multiple industries, including the automotive and aerospace segments. However, the high anisotropy of parts developed through this technique imply that part failure prediction is extremely difficult \textemdash a requirement that must be satisfied to guarantee the safety of the final user. Application of a Failure Criterion to predict part failure has been shown to constitute a solution to this problem. However, specialized printing equipment, and a large number of mechanical tests performed under a variety of loading conditions are required to populate the parameters of the failure function - a process that is extremely time consuming and can prove unfeasible if off-axis printing solutions are not available to the user. This research proposal describes a method by which certain mechanical properties of an FFF part can be predicted using machine learning methods. Data extracted from an FFF printer fitted with in-line sensors that capture extrusion force and velocity, as well as additional data stemming from $\mu$CT scans, dimensional changes in the filament geometry, and mechanical tests can be used to train a machine learning system that can predict the expected mechanical performance of an FFF part under certain loading conditions. This resource can significantly reduce the time required to produce a failure envelope for FFF parts, as well as allowing a better comprehension of the relationship between process variables and final mechanical properties. Additionally, such resources clear the path for development of intelligent equipment that can detect flaws mid-print and auto-correct based on the expected performance of the part.  
 
\vspace{10mm} %10mm vertical space
\textbf{Keywords:} FFF, FDM, Failure Criteria, Mechanical Testing, Machine Learning.

\vfill %Send copyright notice to bottom of the page
\begin{center}
Copyright~\textcopyright: Gerardo A. Mazzei Capote (2021)

\emph{All rights reserved}	
\end{center}
\end{document}