% data_ex.tex
\documentclass[main.tex]{subfiles}
\begin{document}
\chapter{Data Exploration} \label{ch:data_ex}
\section{Foreword} \label{sec:fw_dataex}

Material Extrusion (ME) is the most prevalent Additive Manufacturing (AM) technology due to the broad range of materials and equipment available, generally attainable at a fraction of the cost of other AM processes. While ME has gradually found niche applications outside of the casual user, including implementations in the automotive and aerospace sectors, the slow print speed of the process represents a major pain point of the technology that limits the scope of its adoption. Additionally, process control relies mostly on temperature readings, and there is room for improvement in terms of variations in volumetric throughput and process fluctuations. For this reason, this study uses a customized ME machine with a built-in force sensor and encoder that allows real time acquisition of force and filament print speed data, generating knowledge aimed at maximizing the print speed of ME and improving the understanding of the underlying process physics that can lead to improved print quality and consistency. Using a variety of printing parameters and two materials, it was found that the trend in the data does not completely agree with the two ME melting models known to the researchers of this study. Other noteworthy conclusions include a quantifiable difference on how higher printing temperatures result in lower necessary filament forces, as well as how two materials being printed at the same temperature result in different force-speed pairings.

\section{Experimental Design} \label{sec:doe_data}
\subsection{Equipment and Materials} \label{ssec:mat_data}
\subsubsection{Printer Setup} \label{sssec:printer}

\end{document}