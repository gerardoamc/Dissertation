% data_ex.tex
\documentclass[main.tex]{subfiles}
\begin{document}
\chapter{Trends in Print Force and Extrusion Speed} \label{ch:data_ex}
\section{Foreword} \label{sec:fw_dataex}

Material Extrusion (ME), also known as Fused Filament Fabrication (FFF), is the most prevalent Additive Manufacturing (AM) technology due to the broad range of materials and equipment available, generally attainable at a fraction of the cost of other AM processes. While ME has gradually found niche applications outside of the casual user, including implementations in the automotive and aerospace sectors, the slow print speed of the process represents a major pain point of the technology that limits the scope of its adoption. Additionally, process control relies mostly on temperature readings, and there is room for improvement in terms of variations in volumetric throughput and process fluctuations. For this reason, this study uses a customized ME machine with a built-in force sensor and encoder that allows real time acquisition of force and filament print speed data, generating knowledge aimed at maximizing the print speed of ME and improving the understanding of the underlying process physics that can lead to improved print quality and consistency. 

In the context of this dissertation, every ML solution begins with a data acquisition and exploration step. This study represents just that, as it allowed the researcher to understand an FFF machine with sensors, develop data filtering techniques, and the deployment of automation protocols that made acquisition of the training and validation data sets of the ML model a lot easier and faster. Finally, and as an added bonus, the results of this study shed some light into the current limitations intrinsic to two of the most widely used melting models for FFF: the Bellini \emph{et al.} model, and the Osswald, Puentes, Kattinger model, as both require measurements of filament force and speed to compare data with theoretical predictions \textemdash a feat that was uncommon at the time of this body of work.

\section{Introduction}

One of the major pain points from the Material Extrusion (ME) process is its relatively slow fabrication speed: while compelling for small batches of parts, ME can't come close to competing with other polymer processing techniques, such as injection molding, which can reproduce a complex part in a matter of seconds, as opposed to hours \cite{Baumers2016, Conner2014, Berman2012}. The cause of the sluggish nature of ME is rooted in heat transfer and the underlying physics of the process: as the thermoplastic filament is pushed through a heated chamber, polymer melt is formed and extruded through an opening at the tip of the nozzle, producing a bead of material that will be used to gradually reproduce the geometry of the desired part. The problem lies partially on the melt formation within the nozzle: its physics and limitations are not completely understood at the time of this writing, making print-speed optimization difficult and a matter of trial and error. Attempts to curb this issue through mathematical modeling of the melt formation exist, analytically derived through transport phenomena and rheology. Two major attempts exist, differing in the key assumptions made regarding how the molten polymer is formed and extruded. The first, proposed by Bellini, Güçeri and Bertoldi in 2004 \cite{Bellini2004}, and the second proposed by Osswald, Puentes and Kattinger in 2018 \cite{OsswaldMelting18} and improved upon by Colón et al. in 2020 \cite{ColonQuintana2020}. A schematic representation of both models can be seen in Figure \ref{fig:melt_model}.

\begin{figure}[!htbp]
	\center
	\includegraphics[width=0.7\linewidth, keepaspectratio]{melt_models}
	\caption{Bellini model (left) and Osswald, Puentes, Kattinger model (right) \cite{Oehlmann2021}.} \label{fig:melt_model}
\end{figure}

The Bellini, Güçeri and Bertoldi model assumes that a melt pool forms as soon as the filament enters the heated chamber, with the solid filament acting as a piston that allows the flow of the molten polymer \cite{Bellini2004}. Using the nomenclature shown in Figure 1, the model calculates the pressure requirements to push the molten polymer through three characteristic sections, labeled I, II, and III on the plot. Section I is approximated as pressure flow through a tube. The second section contains a pressure drop associated with the contraction of the nozzle radius, from $R_I$ to $R_{III}$. Finally, section III is also approximated as pressure flow through a tube. As an example, for section I the volumetric flow rate can be estimated using the following equations:

\begin{equation} \label{eq:plm}
	\eta=m\left(T\right){\dot{\gamma}}^{n-1}
\end{equation}

\begin{equation} \label{eq:vfr_mm}
	Q=\left(\frac{n\pi}{3n+1}\right)\left[-\frac{R_1^{3n+1}}{2mL_I^{1/n}}\right]\Delta p_I^{1/n}=A_f\ U_{sz}
\end{equation}

Where, $\eta,\ \dot{\gamma},\ m,\ n$ represent the viscosity, shear rate, consistency index, and power-law index for a Power-law viscosity model for the polymeric melt, while $Q,\ \Delta,\ p_1,\ R_1,\ A_f$ represent the volumetric flow rate, pressure drop, filament radius, and filament area, respectively. It should be noted that the length $L_I$ is unknown and is a function of the filament movement speed $U_{sz}$. The model implies that the pressure necessary to extrude material through the capillary at the nozzle tip (region III) is considerably larger than the pressure requirements in regions I and II for a traditional 3D printer – where the nozzle diameter is usually in the order of hundreds of micrometers, while the filament diameter is either 1.75 mm or 2.85 mm \cite{ColonQuintana2020,Oehlmann2021}. After equating the Filament force $F_z$ to the following equation, one can manipulate the expressions to see that $U_{sz}$ is proportional to $F_z^{1/n}$. For additional details, the reader is invited to carefully examine the publication by Bellini, Güçeri and Bertoldi \cite{Bellini2004}.

\begin{equation} \label{eq:force_bellini}
	F_z=\Delta PA_f\approx\ \Delta p_{III}A_f
\end{equation}

The Osswald, Puentes, Kattinger model approaches the behavior of polymer extrusion akin to the formation of a melt film of limited thickness, governed by melting with pressure flow removal \cite{OsswaldMelting18, ColonQuintana2020}. In this approximation, the melt film thickness $\delta$ is assumed to be orders of magnitude smaller than the diameter of the filament, and the mathematical relationship that arises from deriving equations rooted in transport phenomena yields the following proportionality. The reader is invited to read the publication by Osswald, Puentes, and Kattinger for the derivation of this mathematical relationship \cite{OsswaldMelting18}.

\begin{equation} \label{eq:force_osswald}
	U_{sz}\ \alpha\ F_z^{1/4}
\end{equation}   

Each model has implications for the maximum print-speed possible and the required force necessary to achieve stable printing conditions, as the requisite for extrusion is that the force applied by the filament must be enough to overcome the pressure drop that occurs within the nozzle. However, experimental results that allow discerning between the two models are scarce, mainly because measuring the force exerted by the filament, or the pressure drop within the nozzle during printing is not trivial. Publications by Go, Schiffres, Stevens, and Hart \cite{Go2017}, as well as Go and Hart \cite{Go2017a} managed to measure filament force as a function of filament feed rate through the use of strain gauges. The results showed that pre-heating the filament before the heated chamber through IR allowed for higher throughputs of material at reduced force requirements. Results also indicated that a higher nozzle temperature would allow for higher feedrates at lower forces. However, these results were obtained from either a modified printer with specialized modifications that allow preheating of the filament prior to entry to the heating element, or an experimental setup built with strain gages. A traditional printer executing a print in real time was not examined. Additionally, Coogan and Kazmer \cite{Coogan2019}, and Chen and Smith \cite{Chen2019b} adapted an in-situ rheometer on the print head of the ME machine with the goal of capturing pressure measurements. Serdeczny et al. \cite{Serdeczny2019} also developed an experimental device where a custom ME printhead was adapted to house a load-cell, that reported live force readings to a DAQ system, with the goal of relating force readings to melt pressure and extrudate swelling. All of these studies accomplished measuring print force or melt pressure, however, these devices were either an approximation of a real printer, or the measurements were acquired during steady state and with extrusion being discharged upon open air \textemdash arguably not completely representative of real printing conditions.
  
This work used a desktop ME machine, with a customized filament force sensor and encoder that allows real time data acquisition of filament force and filament velocity during a continuous print. This print was repeated over a variety of print conditions, as well as using two different materials. The data and derived conclusions can then be used to improve the design of future equipment in ME that allows faster print speeds, as well as allowing further refinements in the melting models available for ME systems, as each would benefit from having real data pertaining to filament force and speed.

\section{Equipment and Materials} \label{ssec:mat_data}

\subsection{Printer Setup} \label{ssec:printer}

For this study, an ME 3D printer (Minilab by FusedForm, Colombia) using a 0.4 mm nozzle and capable of printing 1.75 mm filament was equipped with a customized force sensor and thermistor built into the printhead, as well as an encoder that records the extruded filament length over time. The concentric force sensor was positioned just above the hot end, in a Bowden extruder architecture. These modifications permit recording and visualization of live force, filament speed, and temperature data collected during the printing process, while maintaining the original performance and functionality of the 3D printer. The generated data was collected using an Arduino board sampling at a frequency of 5 Hz, connected to MATLAB for visualization, processing, and logging. The printhead setup and encoder positioning can be seen in detail in Figure \ref{fig:printsetup}. This setup has the added benefit of allowing detection of filament slippage, if present. As shown in Figure \ref{fig:printsetup} the external ring of the force sensor sits on the extruder carrier base, secured by the extruder carrier lid. The modified hot-end sits on the internal ring of the force sensor, secured in place by the hot-end-sensor coupler. A small clearance between the carrier lid and the sensor permits the filament force to be transmitted through the assembly and detected by the DAQ system.

\begin{figure}[!htbp]
	\center
	\includegraphics[width=0.9\linewidth, keepaspectratio]{Forcesetup}
	\caption{Assembly shown on the left. Schematic of sensor placement on the right.} \label{fig:printsetup}
\end{figure}

\subsection{Materials} \label{ssec:materials_dataex}

Two materials were chosen to perform the experiments pertaining to this body of work: a customized ABS filament, extruded in-house, and a commercially available PLA filament, each with a nominal diameter of 1.75 mm. The ABS filament was produced using the SABIC Cycolac™ MG94 material. This is an ABS resin traditionally used for injection molding thin-walled parts, as well as ME filament. With a reported Melt Flow Index of 11.7 g/10 min, it is an ideal resin for both the ME and extrusion processes \cite{sabic2016}. The extrusion setup consisted of a single screw extruder (Extrudex EDN 45X30D, Germany) with 45 mm screw diameter and L/D ratio of 30D. The hot melt was extruded at 205 ºC through a circular die with a 4.2 mm diameter. It was then guided through a pre-skinner into a vacuum-assisted, heated water bath (Conair, USA) to cool the extrudate whilst minimizing void formation. The solidified filament then passes through a 3-axis laser micrometer (LaserLinc, USA) and a belt puller (Conair, USA) in a control loop that allows adjustment of the pull speed to keep the extrudate within specification. The desired filament dimensions were a diameter of 1.75 mm with a tolerance of ±0.02 mm. A schematic of the extrusion setup can be seen in Figure \ref{fig:ex_line}. The PLA filament used was the commercially available "Natural PLA PRO" filament sold by Matterhackers, chosen to minimize the effect of colorants/additives to the composition of the filament. Steps were taken to ensure that all the acquired spools of material came from the same lot as to guarantee that processing conditions during the extrusion process were constant. Figure \ref{fig:rheo} shows the complex viscosity for each material, as measured using a 25 mm parallel plate rheometer and a 1 mm gap \cite{ColonQuintana2020}.

\begin{figure}[!htbp]
	\center
	\includegraphics[width=0.9\linewidth, keepaspectratio]{Extrusion_line}
	\caption{Extrusion line used to produce ABS filament} \label{fig:ex_line}
\end{figure}

\begin{figure}[!htbp]
	\center
	\includegraphics[width=\linewidth, keepaspectratio]{rheology}
	\caption{Shear rate dependency for a) PLA and b) ABS} \label{fig:rheo}
\end{figure}

\section{Design of Experiments} \label{sec:doe_data_ex}

A set of preliminary prints were designed with the goal of exploring the raw data, post processing requirements, and detection of lurking variables. In these experiments, the effect of temperature and print speed upon stable printing conditions in terms of required filament force was observed through the execution of several cylindrical toolpath files, where the printhead movement velocity was changed from 15 mm/s in increments of 5 mm/s every 15 layers, each with a thickness of 0.35 mm. To minimize the effects of varying accelerations during the test, a cylindrical geometry with a radius of 75 mm, printed in continuous helical mode was chosen as the benchmark part, as schematized in Figure \ref{fig:cyl_shak}. This ensures that changes in filament force and velocity stem mostly from the extrusion process and not due to toolpath considerations. To verify the effect of print temperature upon the required extrusion force, each material was printed at three different temperatures: 200, 215 and 230°C for PLA, and 215, 230 and 245°C for ABS. Close attention was paid to variability between prints performed at the same print conditions, as well as the maximum stable print speed for each material-temperature pairing.

\begin{figure}[!htbp]
	\center
	\includegraphics[width=0.6\linewidth, keepaspectratio]{cyl_shakira}
	\caption{Helical cylinder experiment} \label{fig:cyl_shak}
\end{figure}

Once conclusions were drawn from the preliminary set of experiments, the same geometry would be reprinted using a toolpath file that guarantees a comparable number of data points to be collected for each print speed-print temperature-material pairing, as well as utilizing optimal printing conditions and data acquisition considerations. 

\section{Results}\label{sec:res_data_ex}
\subsection{Considerations Derived from Preliminary Tests}

Preliminary tests show that the physical limitation of the setup lies between 20 and 25 N of force acting upon the filament. At this level of force, slippage occurs in the drive wheel mechanism that drives the material towards the nozzle, causing the mass throughput to become discontinuous. This behavior manifested itself on the data as a sudden drop in the measured filament force and was accompanied by an audible click on the driving wheel. Each material-temperature pairing reached this threshold at different print speeds. Once this phenomenon was observed in a recurring manner, data acquisition was stopped to minimize the number of outliers and noise.
 
The characteristics of the ME printing process can be best described using a 2D plot, calculating the filament speed against the force within the nozzle. Filament speed is derived from the length data of extruded filament recorded by an encoder, and should not be confused with the printhead movement speed, which is followed by the machine after interpreting gcode.
 
Figure \ref{fig:unp_data} shows the unprocessed signals of length (bottom), the derived speed (middle) and resulting force (top) of a representative test print, made using PLA printed at 230°C. When looking at the plots of unprocessed signals, noise can be observed in the datasets of force and filament speed. Therefore, a combination of visualization and simple signal processing techniques are required to improve the signal to noise ratio of the datasets.

\begin{figure}[!htbp]
	\center
	\includegraphics[width=\linewidth, keepaspectratio]{unp_signal}
	\caption{Raw signal data length (bottom), derived speed (middle), and force (top).} \label{fig:unp_data}
\end{figure}

In this example, the first 200 seconds of data are eliminated, as these constitute the time necessary to add the brim to the print, which is necessary to stabilize the cylinder. Datapoints above 1800 seconds are removed as well due to slippage of the filament and high variability within the signal. To remove the amplified noise of the derived speed signal, a moving average filter, applied to the length data showed the best results while keeping most of the signal information. Different parameters for the sliding window of length $k$ across neighboring elements of the length signal have been tested and number of 20 has been found most effective for this application, using a total of 7100 points in the dataset. Deriving the processed length data using the moving average with a sliding window $k$ of 20 shows good results on the filament speed signal in comparison with the unprocessed speed signal. This is better illustrated in Figure \ref{fig:pandu_1}. The oscillating nature of the post processed signal is of unknown origin, but could be related to either the duty cycle of the heater, or small variations in the filament tension that occurred as the toolpath reproduced the circular geometry of the helix. Figure \ref{fig:pandu_2} shows the comparison of processed and unprocessed force vs. filament speed data, where the reduction of noise is apparent.

\begin{figure}[!htbp]
	\center
	\includegraphics[width=\linewidth, keepaspectratio]{pandu}
	\caption{Comparison of processed and unprocessed filament speed data.} \label{fig:pandu_1}
\end{figure}

\begin{figure}[!htbp]
	\center
	\includegraphics[width=\linewidth, keepaspectratio]{pandu_2}
	\caption{Comparison of processed and unprocessed filament speed vs. force within the nozzle of an FFF printer. PLA printed at 230°C.} \label{fig:pandu_2}
\end{figure}

\pagebreak

Analysis of the post-processed data yielded the following general observations:
\begin{enumerate}
	\item Increasing the hot end temperature allowed the printer to achieve higher filament velocities at lower forces.
	\item Printing at 215°C proved to be unreliable and hard to reproduce, even at lower printing speeds.
	\item Repeatability of results proved to be challenging. Potential causes were determined to be humidity absorbed by the material due to environmental exposure to moisture, partial clog of the nozzle after prolonged use, and additional filament tension caused by the weight of the filament spool preventing an unwinding motion of the material. 
\end{enumerate}

From the conclusions drawn from the preliminary results, the following adjustments were made to the experiment:

\begin{enumerate}
	\item Filament segments were unspooled prior to any experiment to prevent noise originating from filament tension prior to a spool revolution mid-print.
	\item A nozzle cleaning protocol was established, where the printhead would be disassembled, cleaned, and the brass nozzle would be burned at 500°C for 30 minutes to prevent problems arising from nozzle clogging or material degradation. This procedure was executed every time there was a switch in the print material, or every 10 prints, whichever occurred first.
	\item The materials would be dried for a minimum of 3 hours prior to the start of any experiment. This is to minimize the influence of absorbed humidity upon the final materials. ABS was dried at 80°C and PLA at 50°C.
	\item The toolpath file was modified to ensure that each printhead movement speed possessed the same approximate data collection time, as opposed to the same number of print layers. The selected number of layers per printhead movement speed can be seen in Table \ref{tab:layer_time}.
\end{enumerate}

\begin{table}[!htbp] 
	\renewcommand{\arraystretch}{1.5}
	\centering
	\caption{Number of layers per print head movement speed.}
	\newcolumntype{C}{ >{\centering\arraybackslash} m{0.25\linewidth} }
	\newcolumntype{M}{ >{\centering\arraybackslash} m{0.15\linewidth} }
	\begin{tabular}{C C M M} 
		\toprule
		\textbf{Movement speed $[mm/min]$} & \textbf{Movement speed $[mm/s]$} & 
		 \textbf{Time per layer $[s]$} & \textbf{Number of layers}\\
		\midrule
		900 & 15 & 31.41 & 8\\
		1200 & 20 & 23.56 & 10\\
		1500 & 25 & 18.85 & 13\\
		1800 & 30 & 15.71 & 15\\
		2100 & 35 & 13.46 & 18\\
		2400 & 40 & 11.78 & 20\\
		2700 & 45 & 10.47 & 23\\
		3100 & 50 & 9.42 & 25\\
		\bottomrule
	\end{tabular}
	\label{tab:layer_time}
\end{table}

\subsection{Results from Modified Tests}
Results obtained using the modified printing protocol yielded mostly reproducible data. Starting with the ABS material, 5 separate replications of the experiment produced Filament Speed versus Force plots that aligned well with each other. Figure \ref{fig:ABS_230} displays the collected data on the same plot with a transparency filter. Solid areas indicate a denser data cloud. The general trend appears to show an inflection point around the $2 mm/s$ speed mark.

\begin{figure}[!htbp]
	\center
	\includegraphics[width=\linewidth, keepaspectratio]{ABS230}
	\caption{Alpha plot for 5 trials of ABS printed at 230°C.} \label{fig:ABS_230}
\end{figure}

The same experiment performed using 245°C as the printing temperature produced similar results, although with a larger data spread. The general trend confirms what is intuitive: higher printing temperatures allows the machine to reach higher extrusion velocities at lower force requirements. The cause of the spread of the data is not fully understood: monitoring the average printing temperature for each trial showed variability of up to 6°C between replicates. It is unknown if the variations between each print replication are caused exclusively by the temperature variations. Variability in the average printing temperature between replicates was most likely caused by approaching the printer’s upper printing temperature limit, causing the setup difficulty at sustaining the 245°C set value. These results are summarized in Figure \ref{fig:ABS_245}. Note how once again, around the $2 mm/s$ mark there appears to be a change in the trend of the data. Table \ref{tab:data_ex_temp} displays the recorded average temperature for each trial. 

\begin{figure}[!htbp]
	\center
	\includegraphics[width=\linewidth, keepaspectratio]{ABS245}
	\caption{Alpha plot for 5 trials of ABS printed at 245°C.} \label{fig:ABS_245}
\end{figure}

\begin{table}[!htbp] 
	\renewcommand{\arraystretch}{1.5}
	\centering
	\caption{Average print temperature for ABS 245°C trials.}
	\newcolumntype{C}{ >{\centering\arraybackslash} m{0.25\linewidth} }
	\newcolumntype{M}{ >{\centering\arraybackslash} m{0.15\linewidth} }
	\begin{tabular}{M C} 
		\toprule
		\textbf{Trial} & \textbf{Average Print Temperature $[^\circ C]$}\\
		\midrule
		1 & 241.38\\
		2 & 231.81\\
		3 & 241.23\\
		4 & 251.60\\
		5 & 241.21\\
		\bottomrule
	\end{tabular}
	\label{tab:data_ex_temp}
\end{table}

For the PLA material, the most notable difference in behavior with respect to the ABS plots was observed in the general trend of the data clusters. For PLA printed at 230°C, an inflection point in the behavior of the data was not observed. This is illustrated in Figure \ref{fig:PLA_230}. Additionally, the data clusters did not overlap as consistently as with ABS. It is unknown if this difference stems from material properties, or other factors not taken into consideration for this study, such as fluctuations in the filament diameter.

\begin{figure}[!htbp]
	\center
	\includegraphics[width=\linewidth, keepaspectratio]{PLA230}
	\caption{Alpha plot for 5 trials of PLA printed at 230°C.} \label{fig:PLA_230}
\end{figure}

Comparing both materials shows that the PLA based material generally requires higher forces to achieve comparable filament velocities to its ABS counterpart. This can be seen in Figure \ref{fig:compPLAABS}, where a characteristic cluster of data was selected for each material and plotted on the same graph for comparative purposes. This result is particularly interesting, given that in general the complex viscosity of ABS tends to be either higher or similar to that of PLA when compared at a fixed shear rate and temperature, as can be seen in Figure \ref{fig:rheo}.


\begin{figure}[!htbp]
	\center
	\includegraphics[width=\linewidth, keepaspectratio]{compABSPLA}
	\caption{Comparison of PLA and ABS Force-velocity pairings at a printing temperature of 230°C} \label{fig:compPLAABS}
\end{figure}

Finally, the 245°C PLA prints faced the same issues seen with ABS at 245°C: the experiment resulted in force-velocity data clusters that agreed in trends, but not values.  This can be seen in Figure \ref{fig:PLA245}, where trials 1, 2, and 3 match closely, but trials 4 and 5 are shifted to the left. Analyzing the average printing temperature for each trial reveals variations in the printing temperature, but no conclusion can be reached as to the cause of the shift in the data. Repetitions of the experiment yielded similar results, indicating that there are other factors at play other than the effect of the printing temperature. The average values for each printing temperature are shown in Table \ref{tab:data_ex_temp_2}.

\begin{figure}[!htbp]
	\center
	\includegraphics[width=\linewidth, keepaspectratio]{PLA245}
	\caption{Alpha plot for 5 trials of PLA printed at 245°C.} \label{fig:PLA245}
\end{figure}

\begin{table}[!htbp] 
	\renewcommand{\arraystretch}{1.5}
	\centering
	\caption{Average print temperature for PLA 245°C trials.}
	\newcolumntype{C}{ >{\centering\arraybackslash} m{0.25\linewidth} }
	\newcolumntype{M}{ >{\centering\arraybackslash} m{0.15\linewidth} }
	\begin{tabular}{M C} 
		\toprule
		\textbf{Trial} & \textbf{Average Print Temperature $[^\circ C]$}\\
		\midrule
		1 & 241.38\\
		2 & 231.81\\
		3 & 241.23\\
		4 & 251.59\\
		5 & 241.21\\
		\bottomrule
	\end{tabular}
	\label{tab:data_ex_temp_2}
\end{table}

An important observation that can be derived from these experiments is that the data does not fully agree with the Melt Filled Nozzle model proposed by Bellini, Güçeri and Bertoldi \cite{Bellini2004}, nor the Melt Film Model with shear thinning behavior, as modified by Colón et al. \cite{ColonQuintana2020}. Unfortunately, the physical limitations of the setup do not allow for experimentations at higher temperatures or printing speeds, as the drive wheel of the printer would slip, or the heater in the printer would not be able to sustain the elevated temperature while printing. Figure \ref{fig:meltmcomp} shows how the data for a representative case of ABS printed at 230°C compares to the predictions of the Melt Filled Nozzle (FNM) and Melt Film (MFM) models \cite{Bellini2004, OsswaldMelting18, ColonQuintana2020}. ABS printed at 230°C was chosen for this comparison as it proved to be the most consistent condition in all replications of the experiment. The Melt Film model estimations were calculated using two different melting temperatures ($T_m$): the glass transition temperature of the polymer ($T_g$) and $T_g+40K$, as recommended by Colón et al. during his calculations \cite{ColonQuintana2020}. This was done since ABS is an amorphous polymer and it does not have a sharp solid-melt transition point as in the case of semi-crystalline polymeric materials. This last point represents a pain point for this model, as the equations favor sharp melt transitions. Note how the data approximates in magnitude the MF calculations using a $T_m$ of 145°C. On the other hand, the FN model fails to capture the observed behavior in both trend and magnitude. For more details pertaining to how each model estimation was calculated, please refer to the manuscript by Colón et al., where the MG94 ABS material was used to derive the parameters necessary to fit each model \cite{ColonQuintana2020}. Material properties used for the calculations are shown in Table \ref{tab:mat_prop_models_1}, including the power law model parameters necessary to approximate the shear thinning behavior of the material \cite{ColonQuintana2020}.

\begin{figure}[!htbp]
	\center
	\includegraphics[width=\linewidth, keepaspectratio]{meltcomp}
	\caption{Comparison of experimental data for ABS printed at 230°C with predictions stemming from various ME melting models.} \label{fig:meltmcomp}
\end{figure}

\begin{table}[!htbp] 
	\renewcommand{\arraystretch}{1.5}
	\centering
	\caption{Material properties for ABS MG94 \cite{ColonQuintana2020}.}
	\newcolumntype{C}{ >{\centering\arraybackslash} m{0.25\linewidth} }
	\newcolumntype{M}{ >{\centering\arraybackslash} m{0.15\linewidth} }
	\begin{tabular}{C C} 
		\toprule
		\textbf{Material Property} & \textbf{Value}\\
		\midrule
		Melt density & 954 $kg/m^3$\\
		Solid density & 1060 $kg/m^3$\\
		Specific heat & 1381 $J/kg-K$\\
		Thermal conductivity & 0.17 $W/m-K$\\
		Heat of fusion & 0 $J/kg$\\
		\midrule
		$m$ & $4.977x10^{4} Pa - s$\\
		$n$ & $0.2847$\\
		\bottomrule
	\end{tabular}
	\label{tab:mat_prop_models_1}
\end{table}


\pagebreak
\section{Conclusions}

The experiments presented in this study show that a lot is still not fully understood regarding the underlying physics of the ME process. The differences in behavior shown between the ABS and PLA filaments prove that each material needs to be studied in a case-by-case basis to optimize part build time through maximization of print speed. Using the 230°C printing temperature of this study as an example, ABS can achieve approximately 0.5 mm/s of extra filament print speed at comparable printing conditions for the PLA material, even though their complex viscosities are comparable. Additionally, the trends observed in the data do not fully match any of the more renowned melting models available in literature. However, additional testing is necessary given that higher print temperatures and printing forces could not be achieved due to the limitations of the setup. Future work involving a different filament feeding mechanism, and longer and more potent heating sections within the printhead could lead to more comprehensive force-velocity curves. Additionally, these trials involved “best case scenario” prints that circumvent accelerations, decelerations, and filament retractions — all of which occur in real prints and potentially have tangible implications upon the behavior of the material. 
Ultimately a setup similar to the one used here could not only be used to properly calibrate a print to minimize part build time, but also predict other characteristics, such as mass throughput or even final part properties using resources such as machine learning algorithms. Other potential applications include development of smart systems that can auto correct or cancel failing prints according to the real-time monitoring of the filament speed and applied force.

% Nomenclature introduced in this chapter:
\nomenclature[A]{ME}{Material Extrusion}% 
\nomenclature[A]{FNM}{Melt Filled Nozzle Model}%
\nomenclature[A]{MFM}{Melt Film Model}%

\end{document}