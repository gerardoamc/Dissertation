% data_ex.tex
\documentclass[main.tex]{subfiles}
\begin{document}
\chapter{Data Exploration} \label{ch:data_ex}
\section{Foreword} \label{sec:fw_dataex}

Material Extrusion (ME), also known as Fused Filament Fabrication (FFF), is the most prevalent Additive Manufacturing (AM) technology due to the broad range of materials and equipment available, generally attainable at a fraction of the cost of other AM processes. While ME has gradually found niche applications outside of the casual user, including implementations in the automotive and aerospace sectors, the slow print speed of the process represents a major pain point of the technology that limits the scope of its adoption. Additionally, process control relies mostly on temperature readings, and there is room for improvement in terms of variations in volumetric throughput and process fluctuations. For this reason, this study uses a customized ME machine with a built-in force sensor and encoder that allows real time acquisition of force and filament print speed data, generating knowledge aimed at maximizing the print speed of ME and improving the understanding of the underlying process physics that can lead to improved print quality and consistency. 

In the context of this dissertation, every ML solution begins with a data acquisition and exploration step. This study represents just that, as it allowed the researcher to firstly understand an FFF machine with sensors. Secondly, it allowed development of data filtering techniques, and deployment of automation protocols that made acquisition of the training and validation data sets of the ML model a lot easier and faster. Finally, and as an added bonus, the results of this study shed some light into the current limitations intrinsic to both of the most widely used melting models for ME: the Bellini \emph{et al.} model, and the Osswald, Puentes, Kattinger model, as both require measurements of filament force and speed to compare data with theoretical predictions \textemdash a feat that was uncommon at the time of this body of work.

\section{Equipment and Materials} \label{ssec:mat_data}

For this study, an ME 3D printer (Minilab by FusedForm, Colombia) was equipped with a customized force sensor and thermistor built into the printhead, as well as an encoder that records the extruded filament length over time. The concentric force sensor was positioned just above the hot end, in a Bowden extruder architecture. These modifications permit recording and visualization of live force, filament speed, and temperature data collected during the printing process, while maintaining the original performance and functionality of the 3D printer. The generated data was collected using an Arduino board sampling at a frequency of 5 Hz, connected to MATLAB for visualization, processing, and logging. The printhead setup and encoder positioning can be seen in detail in Figure 2. This setup has the added benefit of allowing detection of filament slippage, if present. The printer uses 1.75 mm filament and a 0.4 mm nozzle. 
\subsection{Printer Setup} \label{sssec:printer}

% Nomenclature introduced in this chapter:
\nomenclature[A]{ME}{Material Extrusion}% 

\end{document}