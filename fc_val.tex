% fc_val.tex
\documentclass[main.tex]{subfiles}
\begin{document}
\chapter{Part Structural Integrity Evaluation through FC} \label{ch:fc_val}
\section{Foreword}
Fused Filament Fabrication (FFF) is arguably the most widely available additive manufacturing technology at the moment. Offering the possibility of producing complex geometries in a compressed product development cycle and in a plethora of materials, it has gradually started to become attractive to multiple industrial segments, slowly being implemented in diverse applications. However, the high anisotropy of parts developed through this technique renders failure prediction difficult. The proper performance of the part, or even the safety of the final user, can't be guaranteed under demanding mechanical requirements. This problem can be tackled through the development of a failure envelope that allows engineers to predict failure by using the knowledge of the stress state of the part. Previous research by the authors developed a failure envelope for ABS based, Fused Filament Fabrication (FFF) parts by use of a criterion that incorporates stress interactions. In the context of this dissertation, the work that follows shows how one can use such failure envelope to predict FFF mechanical part failure within 10\% of the real value, and compares how the prediction that stems from the SSIC is much more accurate than those derived from simpler but more ubiquitous FC.

\end{document}